\chapter{Introduction}
\label{cha:1_intro}
The rapid advancement of digital education technologies (DETs) has significantly transformed higher education, reshaping the learning experience and the management of academic activities within the institutions \cite{haleem_understanding_2022}\cite{lacka_examining_2021}. DETs, which encompass a broad range of software and hardware technologies, have played a central role in improving educational accessibility and addressing global challenges in education \cite{schuetze_digitalization_2024}, representing a powerful instrument to achieve the United Nations Sustainable Development Goal 4 for Quality Education.

However, the increasing digitalization of universities has raised many concerns about data privacy, economic dependencies, and institutional autonomy \cite{fiebig_heads_2023}\cite{komljenovic_rise_2021}. The growing reliance on solutions provided by Big Tech companies has intensified worries about the risk of lock-in and loss of data governance, creating a degree of uncertainty regarding the long-term sustainability of these digital infrastructures \cite{angeli_conceptualising_2022}. In addition, the energy consumption and environmental footprint of digital education platforms bring new challenges in aligning technological innovation with sustainability \cite{lago_framing_2015}.

This thesis aims to improve and facilitate the selection process of digital education technologies. To achieve this, it proposes a comprehensive framework that evaluates DETs across all aspects of sustainability, providing a clearer overall picture that highlights the critical strengths and weaknesses of each technology.

\section{Digitalization in Higher Education}
The phenomenon of digitalization, driven by rapid advancements in information and communication technologies (ICT), has led to the widespread adoption of digital education technologies. The integration of the first DETs, such as Learning Management Systems (LMS) and administrative tools, enabled institutions to digitalize content delivery, facilitate student engagement, online learning e administrative tasks \cite{lacka_examining_2021}. With the rise of internet and mobile computing, digital learning further evolved to incorporate new technologies, such as video conferencing tools, and increasingly relying on cloud-based infrastructures. The rapid shift forced by the pandemic exposed strength and weaknesses of digital education, highlighting its potential to improve accessibility while also revealing technical, economic, and social barriers for students and instructors \cite{schuetze_digitalization_2024}.

The outsourcing trend toward Big Tech cloud ecosystems, driven by financial and operational pressures, has allowed universities to quickly adapt to changing educational needs while benefiting of cost efficiency and technical expertise of industry leaders \cite{komljenovic_rise_2021}. However, this trend raised concerns about the outsourcing of university core services, making institutions ever more dependent on private corporations \cite{angeli_conceptualising_2022}. The reliance on cloud providers has also ethical and environmental implications. While many companies claim to limit their environmental impact, many data centers still rely on non-renewable energy, contributing to environmental pollution. Moreover, the continuous upgrading of DETs accelerates hardware obsolescence, contributing to e-waste production and resource depletion. In addition to the diminished control over their own environmental impact, institutions face also the loss of power over data governance, particularly regarding the collection and use of student data \cite{komljenovic_rise_2021}.

All these factors emphasize the need to adopt a critical approach to digitalization, ensuring that universities can make informed, sustainable, and ethical decisions when selecting their digital education technologies. Despite the potential benefits of DETs, the process of selecting technologies that align with institutional values remains a complex challenge.

% \section{The problem of DETs selection}

\section{Knowledge gaps}
Identified knowledge gaps in digital education technologies are as follows:
\begin{itemize}[noitemsep, topsep=4pt, parsep=0pt, partopsep=0pt]
\item While digital education technologies continue to be introduced in higher education institutions, there is still a lack on standardized processes and guidelines about how to select a sustainable DETs 
\item Even though some sustainability aspects are considered more than others in the actual DET selection processes, it is not clear which dimensions of sustainability are involved in the context of education technologies
\item Research literature on digital education does not focus on evaluating or enhancing sustainability of DET, instead it concentrates more on their impact in higher education
\end{itemize}

\section{Goals}
The main goal of this research is to develop a sustainability assessment framework for digital education technologies in universities. Specifically, the study aims to:
\begin{itemize}[noitemsep, topsep=4pt, parsep=0pt, partopsep=0pt]
\item Define the concept of sustainability in the context of DETs, by integrating the relevant dimension identified from research literature.
\item Provide a structured approach for universities that assists those responsible for selection to assess the sustainability of DETs, before the adoption.
\item Demonstrate and discuss its real-world applicability by applying the framework to a DET.
\end{itemize}
Based on the knowledge gaps and goals listed above, this thesis will attempt to answer the following questions:
\begin{center}
    \textit{\textbf{"How can higher education institutions evaluate the sustainability of digital education technologies?"}} \\
    \textit{\textbf{"How can universities facilitate and structure the DET selection process?"}}
\end{center}

% By achieving these goals, this research contributes to the ongoing discourse on sustainable digital education and offers practical guidelines for institutions seeking to adopt responsible digitalization strategies.

\section{Research methodology}
The methodology employed in this study follows a structured approach. First, the research literature was collected and reviewed to identify key dimensions for DETs. Then, the framework was developed based on the outcome of this review and was finally applied to a real-world case study. 
%on those dimensions, defining indicators and evaluation methods. Finally, the framework was applied to a real-world DET, Overleaf, to assess is sustainability and compliance with the defined indicators. 

\subsection{Research literature}
The literature was collected through various methods. First, the literature recommended by the supervisor was reviewed, covering topics related to sustainability frameworks, digital education technologies, and digitalization. From this initial set of sources, both cited and citing articles were examined to expand the collection. Finally, additional articles were retrieved through search engines and Google Scholar. The search queries included combinations of the following keywords: "education", "digital education", "digital education technology", "sustainability", "framework", "indicators", "energy consumption", "emissions", "metrics", "inclusion", "accessibility", "university", and "data center".

\subsection{Framework development}
The framework was developed based on the key sustainability dimensions identified in the literature review. These dimensions provided the foundation for defining relevant indicators, metrics, and evaluation methods to assess the sustainability of digital education technologies. 

First, the selected dimensions were analyzed to determine their applicability to DETs. Then, specific indicators were defined for each dimension, ensuring they were measurable and aligned with sustainability principles. Subsequently, evaluation methods were identified, drawing from established assessment techniques in sustainability and digital education research. In cases where no existing methods were suitable, new evaluation approaches and metrics were developed, as some aspects of sustainability in DETs remain under active research.

\subsection{Case study analysis}
To validate the proposed framework, it was applied to a real-world digital education technology: Overleaf, a collaborative LaTeX platform widely used for academic writing. This case study aimed to assess Overleaf’s sustainability by evaluating its performance against the defined indicators.

\section{Report structure}
This thesis is structured as follows:

\begin{itemize}[noitemsep, topsep=4pt, parsep=0pt, partopsep=0pt]
\item Chapter \ref{cha:2_dets-and-sustainability} provides an overview of DETs, their evolution, and their role in sustainability efforts within higher education.
\item Chapter \ref{cha:3_framework} introduces the proposed sustainability framework, defining the key dimensions and indicators used for evaluation.
\item Chapter \ref{cha:4_case_studies} demonstrates the application of the framework through the assessment of Overleaf.
\item Chapter \ref{cha:5_discussion} analyzes the findings and implications, and highlights challenges and limitations of the framework.
\item Chapter \ref{cha:6_conclusion} summarizes key insights and shares suggestions for future research.
% outlines potential applications of the framework in higher education institutions.
\end{itemize}