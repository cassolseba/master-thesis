\chapter{Discussion}
\label{cha:5_discussion}
This chapter discusses the overall results of the assessment, focusing on how the sustainability framework performed and highlighting critical aspects found in the process.     

\section{Summary of the results}
\label{sec:summary_result}
Based on the results shown in Figure \ref{fig:overleaf_result_overview}, the Community Edition scored 4/21, the cloud-based version 4/24, and the Server Pro version 8/21. Overleaf Server Pro achieved the most balanced score across all evaluated dimensions, except for the environmental dimension, which suffers from the non-applicability of many indicators. As expected, the cloud-based version performed best in the economic and technical dimensions, reflecting the reasons why universities often rely on external cloud infrastructures for their services. On the other hand, the Community Edition obtained a low score due to technical shortcomings, which were predictable given its focus on individual users and relatively small work groups.

A key finding, highlighted by the traffic light system, is that when considering equivalent core functionalities, the way the software is deployed does not affect its pedagogical quality. This means that regardless of the chosen version, the impact on the educational experience remains the same. Consequently, once the pedagogical validity of the technology is established, the other dimensions become decisive in determining which version should be preferred. In contrast, the way the software is deployed may affect the social dimension, as cloud-based solutions do not require technical expertise from institutions, and policies applied by providers tend to collect user data intensively.

Another important insight emerging from the results is that when a university relies on its own infrastructure and internal expertise instead of private providers, it can mitigate many of the challenges associated with using external services. For example, in the environmental dimension, the university has almost full control, allowing it to decide on the type of infrastructure to use, the energy sources to rely on, and the implementation of techniques to minimize its environmental impact. The same applies to the social dimension, where the university gains greater control over data and its management. Consequently, the only remaining limitations are the technical qualities of the education technology itself, which are inherently tied to how the technology was developed and represent one of the few factors beyond the institution’s full control.

\section{Scoring system and weight of indicators}
The framework was designed by assigning equal weight to each indicator. This ensures fairness among the developed indicators, but implies that the number of indicators assigned to a specific dimension determines its overall weight. For example, in the cases of Overleaf Community Edition and Server Pro, the non-applicability of the indicators related to the environmental impact of energy and material consumption reduces the influence of the environmental dimension on the final score. This shift lowers its potential impact from \(\pm\)\ 5 points to approximately \(\pm\)\ 2 points. The same principle applies to other dimensions with fewer indicators or in cases where, at the discretion of those in charge of the assessment, an indicator is re-assessed in a different dimension.

This misalignment with the core principles of the framework must be considered, as each dimension is intended to be equally important in assessing sustainability. A future revision of the framework could consider to re-balance the weight of each indicator based on the quantity of indicators within each dimension. This adjustment would ensure that every dimension can influence the final outcome equitably, whether positively or negatively.

Finally, another aspect to consider regarding the scoring system is that, despite the Server Pro version receiving positive evaluations across most of the indicators, its final score remains surprisingly low. This raises some concerns about the point allocation mechanism, which appears to be overly strict. However, certain aspects of sustainability cannot be overlooked.

\section{Dealing with red zones}
A crucial question that arises from the overall results concerns how to deal with "red zones”:
\begin{center}
    \textit{“What actions should be taken if none of the indicators within a given dimension meets the minimum sustainability criteria?”}
\end{center}

As discussed in the literature review in Chapter \ref{cha:2_dets-and-sustainability}, if a sustainability dimension is overlooked, other dimensions will also be affected. This interdependence is evident in the case of Overleaf cloud-based version, where the weaknesses in the environmental dimension negatively impact the social dimension. Specifically, the lack of transparency about the underlying infrastructure and its impact leads to the neglect of the community and its involvement. Similarly, as observed in the results for Overleaf Community Edition, technical limitations affect the economic dimension. Nevertheless, if none of the indicators within a single dimension achieve a positive evaluation, this should be considered a sufficient criterion for excluding the technology from the selection process.

\section{Challenges of multi-dimensionality and limitations}
%As with any research, the results and implications of this study have to be considered within the context of the research limitations. 
The most significant limitation of this study is its multi-dimensionality, which arises from the nature of sustainability and introduces various challenges both in defining the criteria to be considered and in the evaluation process itself. The application of the framework has proven to be a complex and interdisciplinary process, which risks emphasizing the lack of knowledge in disciplines outside the area of those in charge of the process. For this reason, in this study, the evaluation of many indicators was conducted from a high-level perspective.

Huang M., in his research on DET selection processes \cite{huang_building_2023-1}, demonstrates that in some European universities, this process involves a broader working group, ensuring greater expertise across all relevant disciplines.
Moreover, the case study of this thesis represents only one of many possible scenarios. This selection process is often triggered not only when there is an intention to adopt a new technology, but also when it becomes necessary to replace an existing one. While the framework has performed as expected, further case studies are needed to assess its effectiveness comprehensively.
