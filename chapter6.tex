\chapter{Conclusion}
\label{cha:6_conclusion}
The digitalization of universities continues to evolve rapidly, raising concerns about how sustainability is integrated into this process. The selection of education technologies remains an area with little critical attention, highlighting the need for further research to ensure more informed and sustainable choices. This thesis addressed the issue by first defining the sustainability dimensions for digital education technologies and then proposing an evaluation model to be used in the selection process.

\section{Key insights}
The most important findings in the assessment are resumed as follows:
\begin{itemize}[noitemsep, topsep=4pt, parsep=0pt, partopsep=0pt]
    \item The results confirm that the sustainability in digital education technologies is a multi-dimensional issue that extends beyond mere economic considerations. While financial constraints remain a primary factor in technology procurement, institutions must integrate other sustainability dimensions into their decision-making processes. A structured framework, such as the one developed in this study, enables universities to make more informed choices.
    \item The assessment confirms that the way the software is deployed (cloud-based, self-hosted) does not impact pedagogical effectiveness. Once a technology meets the required educational criteria, other factors such as economic, environmental, and social sustainability become the decisive elements in its selection.
    \item Self-hosted solutions provide improved data governance and more control over environmental impact. However, they remain subject to external technical constraints, such as software architecture and update cycles, which may impact their sustainability in the long-term.
    \item The evaluation confirmed the interconnected nature of the sustainability dimensions, as weaknesses in one sustainability dimension can negatively affect others.
\end{itemize}

\section{Future works}
This study, as discussed in the previous chapter, is not without limitations and presents several opportunities for future research and refinement. 

One of the key areas for further development is the application of the proposed framework to a wider range of case studies. While it has been tested on a specific technology, its adaptability to other digital education technologies and different institutional contexts remains to be explored. As an example, applying the framework to case studies involving DET migration, where institutions transition from one system to another, would provide valuable insights into its practical utility and robustness. Once the framework is validated theoretically, it should be implemented in real-world selection processes to assess its effectiveness in guiding decision-making.

Another crucial aspect is the set of indicators used for evaluation. The indicators in this study were derived from the current state of research, meaning they are not exhaustive and could be expanded as the field evolves. During the framework's development, some indicators were discarded due to challenges in their applicability or measurement, such as resistance to decay and network effects. Future research could revisit these aspects, determining whether they hold value in specific contexts or whether alternative indicators could better capture relevant sustainability factors.

In addition, there is a significant opportunity for improvement in the development of metrics, particularly within the environmental dimension. The complexity of assessing the environmental impact of DETs remains a challenge, as many existing methodologies require extensive data collection and analysis, making them difficult to implement in practice. To address this, more intuitive and comparable metrics should be explored. For example, impact labels, similar to energy efficiency labels used for household appliances, could simplify the assessment of the environmental footprint of digital technologies. Easier sustainability assessments could enhance the framework’s accessibility and usability, supporting more informed decision-making in technology selection.

Future research should also investigate ways to improve the scoring system and indicator weighting to ensure that all dimensions of sustainability are equally represented. As noted in the discussion, certain aspects may currently be under weighted or overemphasized, potentially leading to an imbalanced evaluation of DETs.

Ultimately, this study lays the groundwork for a multi-dimensional assessment of sustainability in digital education technologies, but its full potential can only be realized through further validation, refinement, and practical application.

