\chapter*{Abstract} % no number
\label{abtract}

\addcontentsline{toc}{chapter}{Abstract} % add to index

\bigskip

The rapid adoption of Digital Education Technologies (DETs) has transformed higher education, improving accessibility, collaboration, and administrative efficiency. However, universities often lack structured sustainability assessment frameworks, selecting DETs that lead to vendor lock-in, data privacy concerns, environmental impact, and limited institutional control over digital infrastructures. While cloud-based solutions offer convenience and scalability, they may compromise long-term sustainability and autonomy.

This thesis develops a multi-dimensional sustainability framework to evaluate DETs across technical, economic, social, pedagogical, and environmental dimensions. The framework introduces a structured methodology and a traffic-light scoring system to guide decision-makers in selecting sustainable and ethical digital solutions. The assessment is designed to balance trade-offs between cost efficiency, pedagogical value, institutional control, and environmental impact, ensuring a holistic evaluation process.

To validate the framework, it is applied to Overleaf, a collaborative LaTeX platform, comparing its Community Edition, Server Pro, and cloud-based versions. The results highlight key sustainability trade-offs: while the cloud version excels in economic and technical efficiency, it raises concerns about data privacy and environmental impact. Conversely, self-hosted versions provide greater institutional control and autonomy, but require higher technical expertise and resource investment. The assessment also reveals that pedagogical quality remains unaffected by deployment choice, reinforcing the importance of other sustainability dimensions in the selection process.

A critical discussion on scoring limitations and the challenge of multi-dimensionality underscores the need for future refinements, including weighted indicators to ensure a balanced evaluation. The research concludes that if DETs fail entirely in one sustainability dimension, their adoption should be reconsidered, emphasizing the interdependence of sustainability factors.

By providing a structured and adaptable evaluation framework, this study contributes to sustainable digital transformation in higher education, equipping universities with a transparent decision-making tool to align technology adoption with institutional, ethical, and long-term sustainability goals.